%% The MIT License (MIT)
%%
%% Copyright (c) 2015 Daniil Belyakov
%%
%% Permission is hereby granted, free of charge, to any person obtaining a copy
%% of this software and associated documentation files (the "Software"), to deal
%% in the Software without restriction, including without limitation the rights
%% to use, copy, modify, merge, publish, distribute, sublicense, and/or sell
%% copies of the Software, and to permit persons to whom the Software is
%% furnished to do so, subject to the following conditions:
%%
%% The above copyright notice and this permission notice shall be included in all
%% copies or substantial portions of the Software.
%%
%% THE SOFTWARE IS PROVIDED "AS IS", WITHOUT WARRANTY OF ANY KIND, EXPRESS OR
%% IMPLIED, INCLUDING BUT NOT LIMITED TO THE WARRANTIES OF MERCHANTABILITY,
%% FITNESS FOR A PARTICULAR PURPOSE AND NONINFRINGEMENT. IN NO EVENT SHALL THE
%% AUTHORS OR COPYRIGHT HOLDERS BE LIABLE FOR ANY CLAIM, DAMAGES OR OTHER
%% LIABILITY, WHETHER IN AN ACTION OF CONTRACT, TORT OR OTHERWISE, ARISING FROM,
%% OUT OF OR IN CONNECTION WITH THE SOFTWARE OR THE USE OR OTHER DEALINGS IN THE
%% SOFTWARE.

% The font could be set to Windows-specific Calibri by using the 'calibri' option
\documentclass[]{mcdowellcv}

% For mathematical symbols
\usepackage{amsmath}
\usepackage{color}
\usepackage{academicons}

% Set applicant's personal data for header
\name{Mohammad Ishtiaq Ashiq Khan}
% \address{1209E University Terrace \linebreak Blacksburg VA 24060}
\info{
	\faPhone ~ (540) 449-8267 ~~
	\faAt ~ iashiq5@vt.edu ~~
	\faMapMarker ~ Blacksburg, VA ~~
	\href{https://ashiq5.github.io/}{\faHome ~ Homepage} ~~
	\href{https://www.linkedin.com/in/ishtiaq-ashiq/}{\faLinkedin ~ LinkedIn} ~~
	\href{https://github.com/Ashiq5}{\faGithub ~ GitHub} ~~
	\href{https://scholar.google.com/citations?user=X4jtTKkAAAAJ&hl=en}{\aiGoogleScholar ~ Google Scholar}
}

\begin{document}

	% Print the header
	\makeheader
	
	% Print the content
	\begin{cvsection}{Education}
		\begin{cvsubsection}{Blacksburg, VA}{}{Jan 2021 - Dec 2025 (Expected)}
			\begin{itemize}
				\item Ph.D. in Computer Science and Applications at \textbf{Virginia Tech}; Available for internship during May '24 to Aug '24 %, CGPA: 3.85
				%\item Graduate Coursework: Software Foundations; Computer Architecture; Algorithms; Artificial Intelligence; Comparison of Learning Algorithms; Computational Theory.
				%\item Undergraduate Coursework: Operating Systems; Databases; Algorithms; Programming Languages; Comp. Architecture; Engineering Entrepreneurship; Calculus III.
			\end{itemize}
		\end{cvsubsection}
		\begin{cvsubsection}{Dhaka, Bangladesh}{}{Jul 2014 - Oct 2018}
			\begin{itemize}
				\item B.Sc. in Computer Science at \textbf{Bangladesh University of Engineering and Technology (BUET)}, CGPA: 3.83
				%\item Graduate Coursework: Software Foundations; Computer Architecture; Algorithms; Artificial Intelligence; Comparison of Learning Algorithms; Computational Theory.
				%\item Undergraduate Coursework: Operating Systems; Databases; Algorithms; Programming Languages; Comp. Architecture; Engineering Entrepreneurship; Calculus III.
			\end{itemize}
		\end{cvsubsection}
	\end{cvsection}

	\begin{cvsection}{Selected Publications}
		\begin{cvsubsection}{}{}{}
			\begin{itemize}
				\item \textit{RoVista: Measuring and Analyzing the Route Origin Validation in RPKI} in \textbf{Internet Measurement Conference 2023}.
				\begin{itemize}
					\item \textit{Authors:} Weitong Li, Zhexiao Lin, \textbf{Md. Ishtiaq Ashiq}, Emile Aben, Romain Fontugne, Amreesh Phokeer, Taejoong Chung.
					\item Proposed a network measurement framework, RoVista, to determine the Route Origin Validation status at scale.
				\end{itemize}
				\item \textit{You’ve Got Report: Measurement and Security Implications of DMARC Reporting} in \textbf{USENIX Security 2023}.
				\begin{itemize}
					\item \textit{Authors:} \textbf{Md. Ishtiaq Ashiq}, Weitong Li, Tobias Fiebig, and Taejoong Chung.
					\item Analyzed the DMARC Reporting landscape longitudinally and empirically. Proposed a couple of DoS vulnerabilities
					in 3 major email providers with amplification factor over 1400x leveraging DMARC and TLS-RPT reporting.
				\end{itemize}
				\item \textit{Measuring TTL Violation of DNS Resolvers in the Wild} in \textbf{Passive and Active Measurement 2023}.
				\begin{itemize}
					\item \textit{Authors:} Protick Bhowmick, \textbf{Md. Ishtiaq Ashiq}, Casey Deccio, and Taejoong Chung.
					\item Designed the measurement infrastructure and APIs for the DNSSEC experiment using Docker and Django Rest.
				\end{itemize}
				\item \textit{Under the Hood of DANE Mismanagement in SMTP} in \textbf{USENIX Security 2022}.
				\begin{itemize}
					\item \textit{Authors:} Hyeonmin Lee, \textbf{Md. Ishtiaq Ashiq}, Moritz Muller, Roland van Rijswijk-Deij, Taekyoung Kwon, and Taejoong Chung.
					\item Automated the DANE key rollover scheme in a popular open-source email provider.
				\end{itemize}
				\item \textit{Measurement and Analysis of Automated Certificate Reissuance} in \textbf{Passive and Active Measurement 2021}.
				\begin{itemize}
					\item \textit{Authors:} Olamide Omolola, Richard Roberts, \textbf{Md. Ishtiaq Ashiq}, Taejoong Chung, Dave Levin, and Alan Mislove.  % (2nd author).
					\item Examined SSL certificates issued by leading CAs to identify certificate misissuances based on CAA records.
				\end{itemize}
			\end{itemize}
		\end{cvsubsection}
	\end{cvsection}

	\begin{cvsection}{Selected Projects}
		\begin{cvsubsection}{}{}{}
			\begin{itemize}
				\item \textbf{Revisiting the NXNS Attack} (2022). Developed a scalable technique to measure patches for the attack in local resolvers leveraging a proxy network, \href{https://drive.google.com/file/d/1kuTSIHuNUYxmIKbR6MsSg3znMqx55mCP/view}{[details]}.
				\item \textbf{Transferability of Adversarial Training in Text Domain} (2021). Conducted a study to check transferability of adversarial training across popular adversarial frameworks. Framework: PyTorch, \href{https://github.com/Ashiq5/AdvTrainingExperiment}{[Link]}.
				\item \textbf{DNSSEC Debugger} (2021). Analyzed historical DNSViz data to understand the challenges for DNS administrators while deploying and managing DNSSEC. Presented in \textbf{36th DNS-OARC Workshop}, \href{https://indico.dns-oarc.net/event/40/contributions/891/attachments/857/1555/DNS-OARC-final.pdf}{[Link]}.
				\item \textbf{Robustness Analysis of a Web Honeypot} (2021). Demonstrated common web vulnerabilities in a popular web honeypot framework (SNARE-TANNER), \href{https://drive.google.com/file/d/1-FuDy-8xTE2TCRVV2RDuusnhAiLdgMeO/view?usp=sharing}{[details]}.
			\end{itemize}
		\end{cvsubsection}
	\end{cvsection}

	% \newpage
	\begin{cvsection}{Languages and Technologies}
		\begin{cvsubsection}{}{}{}
			\begin{description}
				\item[Languages] Python, Java, C++, C, JavaScript, HTML, CSS, Assembly (x86), familiar with R and Go
				\item[Frameworks and Technologies] Django Rest, Tensorflow, Apache Spark, PyTorch, Node.js, Celery, Redis, AWS
				\item[DBMS] Oracle SQL, PostgreSQL, MongoDB
				% \item[VCS] GitHub, GitLab
				\item[Tools] Docker, Vagrant, Hugo, etc. %  Munin, Nagios, queryperf, zDNS, BIND, Postfix
			\end{description}
		\end{cvsubsection}
	\end{cvsection}

	\begin{cvsection}{Experience}
		\begin{cvsubsection}{Graduate Research Assistant}{Virginia Tech}{Jan 2021 - Present}
			\begin{itemize}
				% \item Curating experiments, designing APIs, and analyzing big data to identify mismanagements and vulnerabilities in applications of Public Key Infrastructure, DNS, and E-mail.
				\item Conducting data-driven research aimed at enhancing management and security of \textbf{Email}, \textbf{DNS}, and \textbf{PKI}.
				% \item Have multiple publications in top security and measurement venues like USENIX Security, IMC, and PAM.
				% \item Stack: PySpark, MongoDB, PostgreSQL, Node.js, Docker, Django Rest, Redis, AWS, etc.
			\end{itemize}
		\end{cvsubsection}

		\begin{cvsubsection}{Lecturer}{United International University}{Jul 2019 - Dec 2020}
			\begin{itemize}
				\item Taught Network Security, Data Structure, Object-Oriented Programming, etc. undergraduate courses.
			\end{itemize}
		\end{cvsubsection}
		\newpage
		\begin{cvsubsection}{Full Stack Software Engineer}{InfoSapex Limited}{Nov 2018 - Jul 2019}
			\begin{itemize}
				\item Successfully released a Procurement Management System in production with over 50\% contribution.
				% \item Designed a distributed database schema
				\item Served as a technical point of contact with clients and carried out requirement analysis.
				\item Significantly reduced server provisioning time by automating configurations with Puppet and recovery time by setting up monitoring service with Munin and Nagios.
				%\item Managed server-side configuration, monitoring, and deployment, resulting in a 50\% reduction in mean time-to-recovery.
				% \item Stack: Django Rest, Node.js, jQuery, HTML, CSS, Bootstrap, PostgreSQL, Celery
			\end{itemize}
		\end{cvsubsection}

%		\begin{cvsubsection}{Software Engineer Intern}{Dingi Technologies Limited}{Mar 2018 - Aug 2018}
%			\begin{itemize}
%				\item Developed an interactive website for addressing feedbacks from users of Dingi's native mobile application.
%			\end{itemize}
%		\end{cvsubsection}
	\end{cvsection}

	\begin{cvsection}{Additional Experience and Awards}
		\begin{cvsubsection}{}{}{}
			\begin{itemize}
				\item \textbf{Instructor, Virginia Tech:} Taught Intermediate Software Design course during Summer 2023.
    			\item \textbf{Open Source Contributions:} Contributed to 3 open-source projects: \href{https://github.com/mail-in-a-box/mailinabox}{Mail-in-a-Box}, \href{https://github.com/iredmail/iRedAPD}{iRedAPD}, and \href{https://github.com/QData/TextAttack}{TextAttack}.
				\item Awarded \emph{University Merit List Scholarship}, and \emph{Dean's List Scholarship} during bachelor's.
			\end{itemize}
		\end{cvsubsection}
	\end{cvsection}
	% \let\clearpage\relax
\end{document}

